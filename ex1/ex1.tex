% !TEX encoding = UTF-8 Unicode
\documentclass[12pt]{article}
\usepackage[english,greek]{babel}
\usepackage[utf8x]{inputenc}
\usepackage{amssymb,latexsym,amsmath,ucs,amsthm,setspace,graphicx}
\usepackage{kerkis}
\usepackage{tikz}
\usepackage{algorithm2e}

\newcommand{\HRule}{\rule{\linewidth}{0.5mm}}


\begin{document}
\begin{titlepage}
\centering

\includegraphics[scale=0.3]{pyrforos.jpg}

\textsc{\LARGE Σχολή \\ Ηλεκτρολόγων Μηχανικών \\[-3pt] και \\[6pt] Μηχανικών Υπολογιστών}

\vspace{1cm}

% Title
\HRule \\[0.4cm]
{\huge \bfseries Λειτουργικά Συστήματα\\} %\\

\vspace{0.4cm}
\HRule \\[0.4cm]

\Large{1η σειρά ασκήσεων}



% Authors
\vfill
\begin{center}
\large
%Add authors one after another with the same format

Ομάδα \textlatin{oslabb12} 

Γκούμας Βασίλης ---  03113031 

Ζαρίφης Νικόλαος --- 03112178

%\vfill
\end{center}

\end{titlepage}


   

\section*{'Ασκηση 1}

Για τη μεταγλώττιση του κώδικα χρησιμοποιήσαμε το \textlatin{Makefile} που βρίσκεται στον φάκελο 1.1. Στον παραπάνω φάκελο και περιλαμβάνονται όλα τα αρχεία κώδικα που φτιάξαμε.  Τρέχοντας απλά την εντολή \textlatin{make}  γίνεται η μεταγλώττιση και η σύνδεση που δημιουργεί τα δύο εκτελέσιμα. Η έξοδος του δοσμένου εκτελέσιμου είναι \textlatin{Hello oslabb12!}, ενώ αυτού που φτιάξαμε εμείς είναι \textlatin{Welcome to the machine, oslabb12!}

\subsection*{Ερώτημα 1}
 Η επικεφαλίδα είναι ένα αρχείο κώδικα που περιέχει δηλώσεις συναρτήσεων
 χωρίς την υλοποίηση τους.Αφενός χρησιμοποιείται ως \textlatin{API}, ξεχωρίζοντας τη λειτουργικότητα απο την υλοποίηση και αφετέρου μειώνει το \textlatin{compilation/build time} καθώς δε χρειάζεται να γίνουν \textlatin{recompile} τα πάντα μετά απο κάθε αλλαγή. Τέλος προσφέρει οργάνωση και \textlatin{modularity}.
 

 \subsection*{Ερωτήματα 2-3} 
 Το \textlatin{Makefile} που συμπεριλαμβάνεται χρησιμοποιήθηκε για τη δημιουργία του εκτελέσιμου.
 Σημειώνουμε πως για τα αρχεία με κατάληξη \textlatin{*.o}  δεν χρειάζεται να δωθούν οδηγίες πως να φτιαχτούν, καθώς η $make$ απο μόνη της ξέρει να τα φτιάχνει αρχεία κοιτώντας για τα ομώνυμα \textlatin{*.c}. Επίσης το \textlatin{Makefile} περιέχει και το \textlatin{target} του 2ου εκτελέσιμου, με όνομα \textlatin{zing2} . Ο κώδικας για τη δηιουργία του δεύτερου εκτελέσιμου βρίσκεται στα \textlatin{zing2.c , zing2.h}. Για την παραλλαγή χρησιμοποιήθηκε ο δείκτης σε έναν \textlatin{static buffer} που επιστρέφει η \textlatin{getlogin}.
 
 
 \subsection*{Ερώτημα 4}
 Μια λύση στο πρόβλημα είναι να βάλουμε σε διαφορετικά αρχεία τις υλοποιήσεις των συναρτήσεων που είναι πιθανό να αλλάζουν συχνά και να γράψουμε κατάλληλα το \textlatin{Makefile} ώστε κάθε φορά να γίνονται  \textlatin{recompile} μόνο οι συναρτήσεις που άλλαξαν.
 
 
 \subsection*{Ερώτημα 5}
 Το πρόβλημα με τη παραπάνω εντολή είναι ότι το όνομα του εκτελέσιμου ορίστηκε ως \textlatin{foo.c} κάνοντας \textlatin{overwrite} το αρχείο με τον κώδικα.
 
 
 \section*{Άσκηση 2}
 
 \subsection*{Ερώτημα 1} 
 
 Ο κώδικας καθώς και τα αποτελέσματα της \textlatin{strace} βρίσκονται στο φάκελο \textlatin{1.2}
 
 Το \textlatin{output} της \textlatin{strace} πάνω στην \textlatin{fconc} που σχετίζεται με τον κώδικα που γράψαμε βρίσκεται στο αρχείο \textlatin{mystrace} και υλοποιεί τη διαδικασία αντιγραφής. Το υπόλοιπο κομμάτι απο το αποτέλεσμα της \textlatin{strace} βρίσκεται για λόγους πληρότητας στο αρχείο \textlatin{systemstrace}. Χρησιμοποιεί κυρίως την κλήση συστήματος \textlatin{mmap}, η οποία μεταφέρει κομμάτια μνήμης, με σκοπό να φέρει στη σωστή θέση την βιβλιοθήκη της \textlatin{C} και να δημιουργήσει επιτυχώς το \textlatin{runtime} που χρειάζεται το εκτελέσιμο.
 
 Επίσης σχολιάζουμε ότι με \textlatin{input} του προγράμματος \textlatin{A.txt B.txt A.txt} το A περιέχει πλέον τα περιεχόμενα του αρχείου β. Αυτό είναι εντελώς αναμενόμενο, καθώς λόγω της σημαίας  \textlatin{O\_TRUNC} το αρχείο Α μόλις ανοίγεται για εγγραφή διαγράφονται τα προηγούμενα περιεχόμενα του, με αποτέλεσμα το πρώτο όρισμα να μην αντιγράψει τίποτα. Στη συνέχεια αντιγράφεται το Β, μόνο του οποίου τα περιεχόμενα βρίσκονται στο Α.
 Αυτό αποφεύγεται τροποποιώντας ελάχιστα το πρόγραμμα ώστε να βγάζει μήνυμα λάθους αν ένα απο τα αρχεία εισόδου ταυτίζεται με κάποιο αρχείο εξόδου.Μια άλλη λύση είναι η εγγραφή των κειμένων σε ένα προσωρινό αρχείο και μετονομασία του στο σωστό αρχείο, μετά το πέρας της διαδικασίας. 
 
 
 
 \section*{'Ασκηση 3}
 
 \subsection*{Ερώτημα 1} Το \textlatin{output} της \textlatin{strace} βρίσκεται στο αρχείο \textlatin{out}, στο φάκελο \textlatin{1.3}. Όπως βλέπουμε, αρχικά η \textlatin{strace} κάνει \textlatin{fork} τον εαυτό της και τον εκτελεί μέσω της \textlatin{execve system call}
 
 \subsection*{Ερώτημα 2ο}
 Η διαφορά στον παραγώμενο κώδικα  \textlatin{assembly} μεταξύ του  \textlatin{object file}   \textlatin{main.o}  και του εκτελέσιμου \textlatin{zing} είναι ότι το εκτελέσιμο περιέχει μια απόλυτη διεύθυνση ( \textlatin{virtual} )μνήμης ενώ το  \textlatin{ main.o } ένα σχετικό άλμα. Αυτό είναι αναμενόμενο καθώς ο  \textlatin{linker} είναι αυτός που θα κάνει  \textlatin{resolve} όλες τις διευθύνσεις των  \textlatin{object files} και θα δημιουργήσει εκτελέσιμο έτοιμο να τρέξει στην (εικονική) μνήμη. Το εκτελέσιμο έχει ήδη περάσει απο τον \textlatin{linker} ενώ το \textlatin{object file} όχι ακόμα.
 
 
  \subsection*{Ερώτημα 3}
  
  Το πρόγραμμα μας, όπως βρίσκεται στο αρχείο \textlatin{fconc.c} είναι ήδη τροποποιημένο ώστε να υποστηρίζει αόριστο αριθμό παραμέτρων. Η μόνη επιπλέον προσθήκη που χρειάστηκε είναι ένα \textlatin{for loop} για τα ορίσματα και κατάλληλες περιπτώσεις για τo αρχεία εξόδου.
  
  \subsection*{Ερώτημα 4}
  Τρέχοντας το εκτελέσιμο, με τη βοήθεια της \textlatin{strace} διαπιστώνουμε πως το πρόγραμμα προσπαθεί να ανοίξει για διάβασμα το αρχείο \textlatin{/etc/shadow} , αρχείο που παραδοσιακά στα \textlatin{unix} συστήματα περιέχει τα \textlatin{hash} των κωδικών των χρηστών. Αυτό όμως δε γίνεται καθώς το πρόγραμμα δεν έχει τα κατάλληλα δικαιώματα και επομένως τερματίζει με το μήνυμα \textlatin{Problem!}.
  

 
 
 
 
 
 
 








































\end{document}

